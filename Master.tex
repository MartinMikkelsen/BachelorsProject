%Preamble_bachelor2
\documentclass[12pt,openany,twoside,a4paper]{memoir}
\usepackage{graphicx}
\usepackage[margin=2cm]{geometry}
\usepackage[english]{babel}
\usepackage[utf8]{inputenc}
\usepackage[T1]{fontenc}
\usepackage{amsmath}
\usepackage{amssymb}
\usepackage{booktabs}
\usepackage{fancyhdr}
\usepackage{booktabs}
\usepackage{gensymb}
\usepackage{xcolor}
\usepackage{xparse}
\usepackage{float}
\usepackage{bm}					
\usepackage{color} 
\usepackage{fancyvrb} 
\usepackage{lastpage}
\usepackage[multidot]{grffile}
\usepackage{pgf,tikz,pgfplots}
\usetikzlibrary{patterns}
\usetikzlibrary{shadows.blur}
\usetikzlibrary{shapes}
\usepackage[hidelinks]{hyperref}
\usepackage{mathrsfs}
\usepackage{titlesec}
\usepackage{lipsum}
\usetikzlibrary{arrows}
\usepackage{lipsum}
\usepackage[normalem]{ulem}
\usepackage{microtype}
\microtypesetup{final}
\usepackage{url}
\usepackage{xspace}
\usepackage[numbered,framed]{matlab-prettifier}
\usepackage{libertine}            							% Linux Libertine as text font
\usepackage{libertinust1math}     					  % Math support for Linux Libertine
\usepackage[scaled=.95]{newtxtt}  					% Pretty teletype in correct size

\selectfont
\setlrmarginsandblock{3.2cm}{*}{1.5}  				% Old: {3.2cm}{*}{1.5}
\setulmarginsandblock{*}{3.7cm}{1}    				% Old: {*}{3.7cm}{1}
\setlength{\topskip}{1.6\topskip}
\checkandfixthelayout
\sloppybottom
\chapterstyle{southall}
\pagestyle{chapter}
\linespread{1.07}                 								% Due to a strict character-per-page limit
\usepackage[toc,page]{appendix}
\renewcommand\cftappendixname{\appendixname~}



\usepackage{natbib}
\bibliographystyle{apalike}

\setcitestyle{open={[},authoryear,close={]}}
\setcitestyle{aysep={}} 

\DeclareMathOperator{\differential}{d}		
\renewcommand{\vec}[1]{\bm{\mathrm{#1}}}						% Standard vektor.
\newcommand{\uvec}[1]{\hat{\bm{\mathrm{#1}}}}					% Enhedsvektor.
\newcommand{\scalarprod}{\bm{\cdot}}							% Skalarprodukt.
\newcommand{\innerprod}[2]{\left\langle #1,#2 \right\rangle}	% Indre produkt.
\newcommand{\norm}[1]{\left\lVert#1\right\rVert}				% Norm.
\newcommand{\abs}[1]{\left\lvert #1 \right\rvert}				% Absolut værdi.
\newcommand{\MATLAB}{\textsc{Matlab}\xspace}

\NewDocumentCommand{\diff}{O{} m}{{\differential}^{#1} \mkern-1mu #2}	%Spacing after d is half a \! 
\NewDocumentCommand{\difffrac}{O{} m m}{\frac{\diff[#1]{#2}}{\diff{#3}^{#1}}}
\NewDocumentCommand{\partfrac}{O{} m m}{\frac{\partial^{#1}{#2}}{\partial{#3}^{#1}}}
\NewDocumentCommand{\integral}{O{} O{} O{} m}{\int_{#1}^{#2} \diff[#3]{#4} \;}
\newcommand{\projecttitle}{Kinematic invariants to illuminate 3 particle decays}
\begin{document}

% The actual front page
\begin{titlingpage}
	\newlength{\frontpagecorrection}
	\calccentering{\frontpagecorrection}
	\begin{adjustwidth*}{\frontpagecorrection}{-\frontpagecorrection}
		
		\centering
		\sffamily
		
		\vspace*{0.1cm}
		
		\fontsize{22pt}{25pt}\selectfont
		
		\textsc{\projecttitle}
		 \par
		\vspace{0.8cm}
		
		\fontsize{18pt}{22pt}\selectfont
		
		Martin Mikkelsen \\
		201706771 \par
		
		\vspace{1cm}

	
\tikzset{every picture/.style={line width=0.75pt}} %set default line width to 0.75pt        

\begin{tikzpicture}[x=0.75pt,y=0.75pt,yscale=-1,xscale=1]
\path (0,300); %set diagram left start at 0, and has height of 300
\hspace{-2.8cm}
%Straight Lines [id:da9584305228496637] 
\draw [color={rgb, 255:red, 74; green, 144; blue, 226 }  ,draw opacity=1 ]   (4.15,147.5) -- (153.65,147.77) -- (280,148) ;
\draw [shift={(282,148)}, rotate = 180.1] [color={rgb, 255:red, 74; green, 144; blue, 226 }  ,draw opacity=1 ][line width=0.75]    (10.93,-3.29) .. controls (6.95,-1.4) and (3.31,-0.3) .. (0,0) .. controls (3.31,0.3) and (6.95,1.4) .. (10.93,3.29)   ;
%Straight Lines [id:da059582451575133755] 
\draw [color={rgb, 255:red, 208; green, 2; blue, 27 }  ,draw opacity=1 ]   (325.17,130.67) -- (696.75,4.14) ;
\draw [shift={(698.65,3.5)}, rotate = 521.2] [color={rgb, 255:red, 208; green, 2; blue, 27 }  ,draw opacity=1 ][line width=0.75]    (10.93,-3.29) .. controls (6.95,-1.4) and (3.31,-0.3) .. (0,0) .. controls (3.31,0.3) and (6.95,1.4) .. (10.93,3.29)   ;
%Straight Lines [id:da7413765158530913] 
\draw [color={rgb, 255:red, 126; green, 211; blue, 33 }  ,draw opacity=1 ]   (332,148) -- (695.65,147.5) ;
\draw [shift={(697.65,147.5)}, rotate = 539.9200000000001] [color={rgb, 255:red, 126; green, 211; blue, 33 }  ,draw opacity=1 ][line width=0.75]    (10.93,-3.29) .. controls (6.95,-1.4) and (3.31,-0.3) .. (0,0) .. controls (3.31,0.3) and (6.95,1.4) .. (10.93,3.29)   ;
%Straight Lines [id:da03460284354648402] 
\draw [color={rgb, 255:red, 144; green, 19; blue, 254 }  ,draw opacity=1 ][fill={rgb, 255:red, 189; green, 16; blue, 224 }  ,fill opacity=1 ]   (319.5,170) -- (696.11,297.53) ;
\draw [shift={(698,298.17)}, rotate = 198.71] [color={rgb, 255:red, 144; green, 19; blue, 254 }  ,draw opacity=1 ][line width=0.75]    (10.93,-3.29) .. controls (6.95,-1.4) and (3.31,-0.3) .. (0,0) .. controls (3.31,0.3) and (6.95,1.4) .. (10.93,3.29)   ;
%Shape: Circle [id:dp772334667772015] 
\draw  [fill={rgb, 255:red, 0; green, 0; blue, 0 }  ,fill opacity=1 ] (282,148) .. controls (282,134.19) and (293.19,123) .. (307,123) .. controls (320.81,123) and (332,134.19) .. (332,148) .. controls (332,161.81) and (320.81,173) .. (307,173) .. controls (293.19,173) and (282,161.81) .. (282,148) -- cycle ;
%Curve Lines [id:da4602541945701941] 
\draw    (511.91,67.08) .. controls (537,109.5) and (526.67,152.5) .. (525.33,147.5) ;
%Curve Lines [id:da7835201852329977] 
\draw    (525.33,147.5) .. controls (526.33,148.5) and (542.33,200.5) .. (508.57,233.75) ;

% Text Node
\draw (538,95) node [anchor=north west][inner sep=0.75pt]  [font=\Large]  {$s_{1}$};
% Text Node
\draw (539,185) node [anchor=north west][inner sep=0.75pt]  [font=\Large]  {$s_{2}$};
% Text Node
\draw (132,120) node [anchor=north west][inner sep=0.75pt]  [font=\Large]  {$p$};
% Text Node
\draw (380,75) node [anchor=north west][inner sep=0.75pt]  [font=\Large]  {$p_{1}$};
% Text Node
\draw (412,120) node [anchor=north west][inner sep=0.75pt]  [font=\Large]  {$p_{2}$};
% Text Node
\draw (376,200) node [anchor=north west][inner sep=0.75pt]  [font=\Large]  {$p_{3}$};


\end{tikzpicture}

			
		%\includegraphics[width=1\textwidth]{FinalKinematics.pdf}%{Figures/AUSEGL/BLAA/auseglblaa.eps}
		\vspace{1cm}
		
		Bachelor's project \\
		Supervisors: 
		\vspace{1.0cm}
		
		\fontsize{13pt}{14pt}\selectfont
		
		Department of Physics and Astronomy\par
		Aarhus University\par
		Denmark
		
		\vspace{0.3cm}
		
		June 2020
		
	\end{adjustwidth*}
\end{titlingpage}


% ~~~~~~~~~~~~~~~~~~~~~~~~~~~~~~~~~~~~~~~~~~~~~~~~~~~~~~~~~~~~~~~~~~~~~~~~~~~~~
% The verso of the title page
% ~~~~~~~~~~~~~~~~~~~~~~~~~~~~~~~~~~~~~~~~~~~~~~~~~~~~~~~~~~~~~~~~~~~~~~~~~~~~~

% We actually want the page numbering to start at 1 at the front page in order
% to count pages for the requirements! Easy to disable by out-commenting the line.
\setcounter{page}{2}

% ~~~~~~~~
% Abstract
% ~~~~~~~~
\vspace*{-0.5cm}
\section*{Summary}
\addcontentsline{toc}{chapter}{Summary / Resumé}
\thispagestyle{empty}
 
\vspace{1cm}
\section*{Resumé}

\newpage


% ~~~~~~~~~~~~
% The colophon
% ~~~~~~~~~~~~

% Get font-info
\makeatletter
\edef\fontandleading{\@memptsize.0/\the\baselineskip}
\makeatother
\thispagestyle{empty}
% Push to bottom of page and locally set indents
\strut\vfill
{
	\setlength{\parindent}{0pt}
	\addtolength{\parskip}{.6em}
	
	\begin{center}
		\bfseries\sffamily Colophon
	\end{center}
	
	\small
	
	\textsl{\projecttitle}
	
	\smallskip
	
	Bachelor's project by 
	
	The project is supervised by 
	
	Typeset by the author using \LaTeX{} and the \textsf{memoir} document class. Figures are made with the linspecer function in \MATLAB
	
	Code can be found on \url{https://github.com}
	
	Printed at Aarhus University
}


\newpage
\clearpage
{\addtolength{\beforechapskip}{-2\baselineskip}\tableofcontents*}
%\addcontentsline{toc}{chapter}{\numberline{}Summary / Resumé}
\settocdepth{subsection}


\chapter*{Preface} \label{Preface}
\addcontentsline{toc}{chapter}{Preface}
This project is a theoretical treatment of relativistic kinematics. More specifically the kinematics concerning decays with three particles in the final state. To illustrate three-body decays I introduce the term kinematic invariants and show how these can be related to a phase-space. In other words, a graphical representation of relativistic kinematics and how one can deduce information about the kinematics.
\chapter{Introduction} \label{Introduction}
\vspace{-1cm}


\chapter{Theory} \label{sec:theory}

\chapter{Discussion} \label{sec:discussion}

\bibliography{Bibliography.bib}
\begin{appendices} 
\chapter{}\label{Cosmology}

\end{appendices}
\end{document}